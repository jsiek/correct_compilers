%%
%% This is file `sample-sigplan.tex',
%% generated with the docstrip utility.
%%
%% The original source files were:
%%
%% samples.dtx  (with options: `all,proceedings,bibtex,sigplan')
%% 
%% IMPORTANT NOTICE:
%% 
%% For the copyright see the source file.
%% 
%% Any modified versions of this file must be renamed
%% with new filenames distinct from sample-sigplan.tex.
%% 
%% For distribution of the original source see the terms
%% for copying and modification in the file samples.dtx.
%% 
%% This generated file may be distributed as long as the
%% original source files, as listed above, are part of the
%% same distribution. (The sources need not necessarily be
%% in the same archive or directory.)
%%
%%
%% Commands for TeXCount
%TC:macro \cite [option:text,text]
%TC:macro \citep [option:text,text]
%TC:macro \citet [option:text,text]
%TC:envir table 0 1
%TC:envir table* 0 1
%TC:envir tabular [ignore] word
%TC:envir displaymath 0 word
%TC:envir math 0 word
%TC:envir comment 0 0
%%
%% The first command in your LaTeX source must be the \documentclass
%% command.
%%
%% For submission and review of your manuscript please change the
%% command to \documentclass[manuscript, screen, review]{acmart}.
%%
%% When submitting camera ready or to TAPS, please change the command
%% to \documentclass[sigconf]{acmart} or whichever template is required
%% for your publication.
%%
%%
\documentclass[sigplan,review,dvipsnames,screen,10pt]{acmart}
%\documentclass[sigplan,screen]{acmart}
%%
%% \BibTeX command to typeset BibTeX logo in the docs
\AtBeginDocument{%
  \providecommand\BibTeX{{%
    Bib\TeX}}}

%% Rights management information.  This information is sent to you
%% when you complete the rights form.  These commands have SAMPLE
%% values in them; it is your responsibility as an author to replace
%% the commands and values with those provided to you when you
%% complete the rights form.
\setcopyright{acmlicensed}
\copyrightyear{2018}
\acmYear{2018}
\acmDOI{XXXXXXX.XXXXXXX}
%% These commands are for a PROCEEDINGS abstract or paper.
\acmConference[OlivierFest '25]{Workshop dedicated to Olivier Danvy's 65th birthday}{October,
  2025}{Singapore}
%%
%%  Uncomment \acmBooktitle if the title of the proceedings is different
%%  from ``Proceedings of ...''!
%%
%%\acmBooktitle{Woodstock '18: ACM Symposium on Neural Gaze Detection,
%%  June 03--05, 2018, Woodstock, NY}
\acmISBN{978-1-4503-XXXX-X/2018/06}


%%
%% Submission ID.
%% Use this when submitting an article to a sponsored event. You'll
%% receive a unique submission ID from the organizers
%% of the event, and this ID should be used as the parameter to this command.
%%\acmSubmissionID{123-A56-BU3}

%%
%% For managing citations, it is recommended to use bibliography
%% files in BibTeX format.
%%
%% You can then either use BibTeX with the ACM-Reference-Format style,
%% or BibLaTeX with the acmnumeric or acmauthoryear sytles, that include
%% support for advanced citation of software artefact from the
%% biblatex-software package, also separately available on CTAN.
%%
%% Look at the sample-*-biblatex.tex files for templates showcasing
%% the biblatex styles.
%%

%%
%% The majority of ACM publications use numbered citations and
%% references.  The command \citestyle{authoryear} switches to the
%% "author year" style.
%%
%% If you are preparing content for an event
%% sponsored by ACM SIGGRAPH, you must use the "author year" style of
%% citations and references.
%% Uncommenting
%% the next command will enable that style.
%%\citestyle{acmauthoryear}

\newcommand{\TRUE}[0]{\mathsf{true}}
\newcommand{\FALSE}[0]{\mathsf{false}}
\newcommand{\IF}[3]{\mathsf{if}\,#1\,\mathsf{then}\,#2\,\mathsf{else}\,#3}

\newcommand{\CE}[1]{\mathcal{E} \llbracket #1 \rrbracket}
\newcommand{\CB}[3]{\mathcal{B} \llbracket #1 \rrbracket ( #2, #3 )}

%%
%% end of the preamble, start of the body of the document source.
\begin{document}

%%
%% The "title" command has an optional parameter,
%% allowing the author to define a "short title" to be used in page headers.
\title{Verified Nanopasses for Compiling Conditionals}

%%
%% The "author" command and its associated commands are used to define
%% the authors and their affiliations.
%% Of note is the shared affiliation of the first two authors, and the
%% "authornote" and "authornotemark" commands
%% used to denote shared contribution to the research.
\author{Jeremy G. Siek}
%\authornote{Both authors contributed equally to this research.}
\email{jsiek@iu.edu}
%\orcid{1234-5678-9012}
%% \author{G.K.M. Tobin}
%% \authornotemark[1]
%% \email{webmaster@marysville-ohio.com}
\affiliation{%
  \institution{Indiana University}
  \city{Bloomington}
  \state{IN}
  \country{USA}
}

%% \author{Lars Th{\o}rv{\"a}ld}
%% \affiliation{%
%%   \institution{The Th{\o}rv{\"a}ld Group}
%%   \city{Hekla}
%%   \country{Iceland}}
%% \email{larst@affiliation.org}

%% \author{Valerie B\'eranger}
%% \affiliation{%
%%   \institution{Inria Paris-Rocquencourt}
%%   \city{Rocquencourt}
%%   \country{France}
%% }

%% \author{Aparna Patel}
%% \affiliation{%
%%  \institution{Rajiv Gandhi University}
%%  \city{Doimukh}
%%  \state{Arunachal Pradesh}
%%  \country{India}}

%% \author{Huifen Chan}
%% \affiliation{%
%%   \institution{Tsinghua University}
%%   \city{Haidian Qu}
%%   \state{Beijing Shi}
%%   \country{China}}

%% \author{Charles Palmer}
%% \affiliation{%
%%   \institution{Palmer Research Laboratories}
%%   \city{San Antonio}
%%   \state{Texas}
%%   \country{USA}}
%% \email{cpalmer@prl.com}

%% \author{John Smith}
%% \affiliation{%
%%   \institution{The Th{\o}rv{\"a}ld Group}
%%   \city{Hekla}
%%   \country{Iceland}}
%% \email{jsmith@affiliation.org}

%% \author{Julius P. Kumquat}
%% \affiliation{%
%%   \institution{The Kumquat Consortium}
%%   \city{New York}
%%   \country{USA}}
%% \email{jpkumquat@consortium.net}

%%
%% By default, the full list of authors will be used in the page
%% headers. Often, this list is too long, and will overlap
%% other information printed in the page headers. This command allows
%% the author to define a more concise list
%% of authors' names for this purpose.
%\renewcommand{\shortauthors}{Trovato et al.}

%%
%% The abstract is a short summary of the work to be presented in the
%% article.
\begin{abstract}
We present a proof of correctness in Agda for four nanopasses that
translate a source language with let binding, integer arithmetic,
conditional expressions and Booleans into an x86-flavored register
transfer language. The most interesting of these four nanopasses is a
translation of conditional expressions into goto-style control flow
that uses the continuation-oriented approach of Olivier Danvy's
one-pass transformation into monadic normal form (2003).
\end{abstract}

%%
%% The code below is generated by the tool at http://dl.acm.org/ccs.cfm.
%% Please copy and paste the code instead of the example below.
%%
%% \begin{CCSXML}
%% <ccs2012>
%%  <concept>
%%   <concept_id>00000000.0000000.0000000</concept_id>
%%   <concept_desc>Do Not Use This Code, Generate the Correct Terms for Your Paper</concept_desc>
%%   <concept_significance>500</concept_significance>
%%  </concept>
%%  <concept>
%%   <concept_id>00000000.00000000.00000000</concept_id>
%%   <concept_desc>Do Not Use This Code, Generate the Correct Terms for Your Paper</concept_desc>
%%   <concept_significance>300</concept_significance>
%%  </concept>
%%  <concept>
%%   <concept_id>00000000.00000000.00000000</concept_id>
%%   <concept_desc>Do Not Use This Code, Generate the Correct Terms for Your Paper</concept_desc>
%%   <concept_significance>100</concept_significance>
%%  </concept>
%%  <concept>
%%   <concept_id>00000000.00000000.00000000</concept_id>
%%   <concept_desc>Do Not Use This Code, Generate the Correct Terms for Your Paper</concept_desc>
%%   <concept_significance>100</concept_significance>
%%  </concept>
%% </ccs2012>
%% \end{CCSXML}

%% \ccsdesc[500]{Do Not Use This Code~Generate the Correct Terms for Your Paper}
%% \ccsdesc[300]{Do Not Use This Code~Generate the Correct Terms for Your Paper}
%% \ccsdesc{Do Not Use This Code~Generate the Correct Terms for Your Paper}
%% \ccsdesc[100]{Do Not Use This Code~Generate the Correct Terms for Your Paper}

%%
%% Keywords. The author(s) should pick words that accurately describe
%% the work being presented. Separate the keywords with commas.
\keywords{short-cut boolean evaluation, nanopass, verified compilation, mechanized proof, Agda}
%% A "teaser" image appears between the author and affiliation
%% information and the body of the document, and typically spans the
%% page.
%% \begin{teaserfigure}
%%   \includegraphics[width=\textwidth]{sampleteaser}
%%   \caption{Seattle Mariners at Spring Training, 2010.}
%%   \Description{Enjoying the baseball game from the third-base
%%   seats. Ichiro Suzuki preparing to bat.}
%%   \label{fig:teaser}
%% \end{teaserfigure}

%% \received{20 February 2007}
%% \received[revised]{12 March 2009}
%% \received[accepted]{5 June 2009}

%%
%% This command processes the author and affiliation and title
%% information and builds the first part of the formatted document.
\maketitle

\section{Introduction}

The compilation of Boolean expressions and conditionals is a classic
topic in the compiler literature, going back to an algorithm of
\citet{Aho:1986qf} (presented in the form of an attribute grammar)
that translates Boolean expressions and conditionals into gotos. The
compilation of Boolean expressions and conditionals often occurs at an
interesting stage in the compilation pipeline, where the intermediate
language changes from an abstract syntax \emph{tree} into a control
flow \emph{graph}, or equivalently, an intermediate language with
\texttt{goto}. Further, when targeting an assembly language such as
Intel's x86, the naive approach to compiling boolean expressions and
conditionals produces embarassingly inefficient code. Similar issues
arise in the compilation of languages such as Haskell, where the
case-of-case transformation is employed to optimize dispatching on
algebraic datatypes \citep{PEYTONJONES19983}.

Olivier Danvy, in his paper \emph{A New One-Pass Transformation into
Monadic Normal Form}~\citep{Danvy:2003fk}, presents a beautiful family
of structurally recursive functions that translate Boolean expressions
and conditionals into a monadic normal form that (1) no longer
contains Boolean expressions, (2) the condition expression of a
condition is a value, and (3) jumps are expressed using zero-arity
function calls. One of the key parts of that translation is in the
compilation of a conditional expression such as $(\IF{b}{e_1}{e_2})$,
where the expression $b$ is compiled using a special function
$\mathcal{B}$ that takes two more arguments, the generated code for
the two branches. Thus, $\mathcal{B}$ can analyze the expression $b$
to decide how to generate code for the $\mathsf{if}$, or whether an
$\mathsf{if}$ is even necessary.
\[
\CE{\IF{b}{e_1}{e_2}} = \CB{b}{\CE{e_1}}{\CE{e_2}}
\]
For example, in the simple cases where $b$ is $\TRUE$ or $\FALSE$,
no branching is needed.
\begin{align*}
\CB{\TRUE}{c_1}{c_2} &= c_1 \\
\CB{\FALSE}{c_1}{c_2} &= c_2
\end{align*}

My first exposure to these ideas were in Kent Dybvig's compiler
course~\citep{Dybvig:2010aa} where he taught the techniques used in
the Chez Scheme compiler~\citep{Dybvig:2006aa}.  One of those
techniques is called destintation-driven code
generation~\cite{Dybvig:1990aa}, which adds an extra parameter to the
compilation function to indicate how the context will use the result
value of an expression. The context may (1) ignore the result (and
only care about the expression's side effects), (2) branch based on
interpreting its result as a boolean, (3) assign the result to a
variable, or (4) output the result. These options are encoded as a
disjoint union.  Relating this to Olivier's transformation functions,
the $\mathcal{E}$ function is for contexts that need the actual value
and the $\mathcal{B}$ function is for contexts that branch on the
expression's result. Of course, a pair of functions is isomorphic to a
function on a disjoint union.
\[
(A \to C) \times (B \to C) \cong (A + B) \to C
\]

The CompCert compiler \citep{Leroy:2006fe} (a compiler for C verified
in Coq) uses the one-function per context approach of Danvy in the
RTLgen pass that translates from the CminorSel intermediate language
to CompCert's register transfer language. The verified CakeML
compiler, on the other hand, employs a naive translation of Boolean
expressions and conditionals for the compilation of stackLang the
intermediate language into labLang~\citep{Kumar:2014aa}.

Towards developing a course on verified compilation, I am interested
in adapting verification techniques from CompCert to the compiler
presented in the \emph{Essentials of Compilation}
textbook~\citep{Siek:2023tr,Siek:2023ue}. That book takes an
incremental approach to teaching compilers, that is, it guides the
students through the creation of multiple compilers for a sequence of
languages, each of which adds a language feature to the previous
langauge. For the language with Boolean expressions and conditionals,
the Explicate Control pass of the compiler uses the one-function per
context approach of Danvy to translate into an x86-flavored register
transfer language.  \emph{Essentials of Compilation} also employs the
nanopass approach, breaking up the compiler into many small passes
that are easier for student to digest~\citep{Sarkar:2004fk}.

This paper presents a proof of correctness in Agda for four nanopasses
that translate a source language with let binding, integer arithmetic,
conditional expressions and Booleans into an x86-flavored register
transfer language. These four passes correspond to the
following three passes in \emph{Essentials of Compilation}:
\begin{enumerate}
\item Remove Complex Operands
\item Explicate Control
\item Select Instructions
\end{enumerate}
The process of proving of correctness of the Explicate Control pass
made it clear to me that it has two distinct reponsibilities that
deserve to be separated:
\begin{itemize}
\item Replacing \texttt{let} expressions with assignment statements.
\item Translating Boolean expressions and conditionals into
  \texttt{goto}-based control-flow.
\end{itemize}
(The RTLgen pass of CompCert also handles the above two
responsibilities.)  So in this paper, the Explicate Control pass is
split into two passes, the first named Lift Locals and the second that
keeps the name Explicate Control.
\begin{enumerate}
\item Remove Complex Operands
\item Lift Locals
\item Explicate Control
\item Select Instructions
\end{enumerate}


%% differs in that I also use a big-step semantics for the register
%% transfer language (instead of small step), which streamlines some of
%% the reasoning.



%%
%% The next two lines define the bibliography style to be used, and
%% the bibliography file.
\bibliographystyle{ACM-Reference-Format}
\bibliography{all}


%%
%% If your work has an appendix, this is the place to put it.
%\appendix


\end{document}
\endinput
%%
%% End of file `sample-sigplan.tex'.
